\documentclass{article}
\usepackage{xeCJK}
\usepackage{geometry}
\usepackage{amsmath}
\title{植物物种多样性对植物种群抗旱能力的影响}
\author{郝鹏飞 202111998147 丁天健 202111130013 尚{\CJKfontspec{宋体-简}洺}石 202111998032 }
\date{2023年3月19日}

\begin{document}
\maketitle{}






[[[failure]]]

不同种类的植物对干旱具有不同的反应,大量观察表明植物物种的数量在一个种群面对干旱周期的过程中起着重要作用。本文首先分析了需要解决的问题,随后根据科学知识提出了有关植物种群的假设并给出建模中需使用的符号。为研究长期、较大范围的干旱与物种多样性对植物种群的影响,建立了元胞自动机模型模拟在周期性干旱的条件下种群中各物种数量的变化,最后以图表方式呈现结果。

[[[failure]]]

\section{问题分析}


问题的目标为验证植物多样性在保水抗旱方面的重要意义,具体而言,需要解决以下问题:

[[[failure]]]

对于第一个问题,考虑植物群落在短期内物种种群数量的变化。在植物群落中,温度、水量[[[failure]]] [[[failure]]]对植物的生长以及繁殖有影响。同时,植物种群之间的竞争会对植物种群的增长产生消极影响。我们以[[[failure]]]方程[[[failure]]] \[[[[failure]]]\] 为基础,其中$y$为种群数量,$r$为该种群的增长率,$M$为环境容纳量.建立考虑以上几个因素的的种群数量微分方程模型。

同时,在查阅相关文献后,我们提出了描述群落稳定性的指标[[[failure]]],即以各个物种的种群数量的方差的加权平均值为基础来刻画群落在面对干旱条件时的群落稳定性。

对于第二、三个问题,对生物群落而言,各个物种之间也有合作关系,即一个物种的繁荣能够为另一个物种的生存和繁荣提供基础,并且在面对干旱等灾害时,物种之间的合作能够增加他们度过危机的机会[[[failure]]];同时,如果考虑植物群落在更长时间内的变化,植物群落将会遭受到更多的灾害和经历更多的环境变化;而且各个物种之间存在进化的过程,低级植物为高级植物提供了生长的基本条件;植物的繁殖存在有性繁殖和无性繁殖两种,他们与物种死亡率共同决定了物种的增长率。由于长时间环境变化的复杂性以及随机因素,我们考虑将时间、空间离散化,并通过元胞自动机来展示植物群落与环境的长期作用。

\section{假设与符号}


\subsection{假设与理由}


[[[failure]]]

\subsection{符号}


[[[failure]]]

\section{短期内种群数量变化微分方程模型}


我们首先考虑将方程(1)拓展到多个物种。我们设在该地域共存在$n$个	extbf{物种},分别记为$S_1,S_2,\cdots,S_n$,并用$y_i$代表物种$S_i$的	extbf{种群数量}.根据假设我们知道,多个物种之间的竞争会抑制种群数量的增加,因此我们考虑将方程(1)中的因子$(1-\frac{y}{M})$修改为 [[[failure]]] 我们还知道,不同植物生长需要的水量不同,我们将各个物种对水的需求量分别记为$C_1,C_2,\cdots,C_n$,称为	extbf{需水量}。同时我们用因子 [math frac{\sum_{j=1}^nC_j y_j}{M_0}] 来刻画所有物种生长需要的水量占环境可以提供的总资源的比值,并且设$M_0$为环境可以提供的总资源,称为	extbf{环境容纳量}。将$n$个物种的方程全部列出,我们得到

%由生物学知识,当一个群落中种群数量逐渐上升时,种群数量会抑制增长率,使增长率下降,两种物种也会彼此进行抑制,进行土地资源等生存条件的争夺。其次,我们强调干旱对植物群落的影响,在考虑干旱对植物群落的影响时,我们注意到物种的需水量和吸水能力均会对建模的过程产生影响。为了提高在后续整合多个物种模型时的准确性,我们在此忽略吸水能力在物种资源竞争中所造成的影响,故在原有$Logistic$规律的基础上,我们用多种物种的种群数量与各自需水量的乘积之和$\sum_{j=1}^n\alpha_jy_j$来替代$y$,(此处的需水量是相对于需水量为$1$的物种的绝对需水量) \begin{equation} \left{ \begin{aligned} %
onumber & y_1^{\prime}=r_1y_1(1-\frac{\sum_{j=1}^nC_jy_j}{M})\ & y_2^{\prime}=r_2y_2(1-\frac{\sum_{j=1}^nC_jy_j}{M})\ & \cdots \ & y_n^{\prime}=r_ny_n(1-\frac{\sum_{j=1}^nC_jy_j}{M}) \end{aligned} 
ight. \end{equation}

\begin{figure}[[[failure]]]

\centering \includegraphics[[[failure]]]{pic.jpg}

\caption{流程图} \label{fig:b} \end{figure}

\subsection{参数的选取} 我们以一个月为周期研究群落的变化,即$t_i-t_{i-1}$为一个月。在现实中一株植物所占空间大小可以近似为$1m	imes 1m$,我们将研究的范围取为$100m	imes 100m$,并且近似认为一个格子的大小为$1m	imes 1m$.

首先考虑植物的需水量$C_i$。$C_i$指的是植物生长需水量的多少,一般来说,植物生长需水量并不容易测定,为了便于使用,我们根据植物的生物量来近似得出需水量数据。根据数据中提供的各个植物的生物量,我们估算出了如下几种植物的生物量数据: \begin{align*} \begin{pmatrix} C_1 & C_2 & C_3 & C_4 & C_5 & C_6 & C_7 \end{pmatrix} &= \begin{pmatrix} 1 & 2 & 4 & 10 & 12 & 15 & 20 \end{pmatrix} \end{align*}

将上述数据写在表格中,在接下来的研究中,我们以主要研究表
ef{table:1}中的各个物种.

\begin{table}[[[failure]]]

\begin{center} \begin{tabular}{cccccccc}

	oprule $n$ & 1 & 2 & 3 & 4 & 5 & 6 & 7 \ \midrule $C_n$ & 1 & 2 & 4 & 10 & 12 & 15 & 20 \ \midrule $A_n$ & 2 & 5 & 8 & 20 & 25 & 37 & 50 \ \bottomrule \end{tabular} \caption{物种的年龄及需水量} \label{table:1} \end{center} \end{table}

\subsection{多少物种可以使群落受益}

为了探索群落中物种数目对群落受益的影响,我们以参数

分别计算了$n=1,\ldots,7$时的群落随时间变化的过程,$n=k$是指这个群落中含有物种$S_1,\ldots,S_k$。定义群落中物种数量为$n$时群落的稳定性

其中$D(Si)$ 为物种$i$的数量关于时间的方差,由定义知$W_n$衡量了植物群落中每种植物数量的变化程度,从而可以较为全面地刻画群落的稳定性。易知$W_n$的值越小,即各个物种种群数量越稳定,群落稳定性越高。

对于每个$n$,分别重复运行了五次,得到了如表
ef{tab:result}中$1,\cdots5$的结果,并分别计算其平均值和方差如表。并且得到了如图
ef{fig:b}结果,左图表示群落的空间分布,右图表示各个种群的数量变化。下面仅展示$n=1,4,7$情形下植物种群分布的图像。

\begin{table} \caption{结果} \label{tab:result} \centering \begin{tabular}{cccccccc} 	oprule $n$ & 1 & 2 & 3 & 4 & 5 & 平均值 & 方差 \ \midrule 1 & 6.8523 & 6.8447 & 6.8229 & 6.8197 & 6.8382 & 6.83556 & 0.000195638\ 2 & 6.2520 & 6.2580 & 6.2692 & 6.2902 & 6.2491 & 6.2637 & 0.00027876\ 3 & 6.0154 & 6.0003 & 6.0536 & 6.0335 & 5.9879 & 6.01814 & 0.000683393\ 4 & 5.9699 & 5.9483 & 5.9529 & 5.9921 & 5.9542 & 5.96348 & 0.000322202\ 5 & 5.8185 & 5.9186 & 5.7865 & 5.8719 & 5.8368 & 5.84646 & 0.002580413\ 6 & 5.8501 & 5.8484 & 5.7163 & 5.7589 & 5.7372 & 5.78218 & 0.003975887\ 7 & 5.7551 & 5.7055 & 5.7106 & 5.7675 & 5.7811 & 5.74396 & 0.001162418\ \bottomrule \end{tabular} \end{table}

\begin{figure}

\centering

\caption{模型计算结果} \label{fig:b} \end{figure} 
\end{document}